\documentclass{article}


% if you need to pass options to natbib, use, e.g.:
%     \PassOptionsToPackage{numbers, compress}{natbib}
% before loading neurips_2023


% ready for submission
\usepackage{neurips_2023}


% to compile a preprint version, e.g., for submission to arXiv, add add the
% [preprint] option:
%     \usepackage[preprint]{neurips_2023}


% to compile a camera-ready version, add the [final] option, e.g.:
%     \usepackage[final]{neurips_2023}


% to avoid loading the natbib package, add option nonatbib:
%    \usepackage[nonatbib]{neurips_2023}


\usepackage[utf8]{inputenc} % allow utf-8 input
\usepackage[T1]{fontenc}    % use 8-bit T1 fonts
\usepackage{hyperref}       % hyperlinks
\usepackage{url}            % simple URL typesetting
\usepackage{booktabs}       % professional-quality tables
\usepackage{amsfonts}       % blackboard math symbols
\usepackage{nicefrac}       % compact symbols for 1/2, etc.
\usepackage{microtype}      % microtypography
\usepackage{xcolor}         % colors


\title{A literature review on applications of optimal transport in deep learning : Looking at Wasserstein Distance as a Loss Function }


% The \author macro works with any number of authors. There are two commands
% used to separate the names and addresses of multiple authors: \And and \AND.
%
% Using \And between authors leaves it to LaTeX to determine where to break the
% lines. Using \AND forces a line break at that point. So, if LaTeX puts 3 of 4
% authors names on the first line, and the last on the second line, try using
% \AND instead of \And before the third author name.


\author{%
  Makkunda Sharma \\
  Department of Computer Science\\
  University of Oxford\\
  \texttt{makkunda.sharma@cs.ox.ac.uk} \\
}


\begin{document}
 

\maketitle


\begin{abstract}
 This is a literature review looking at recent use of wasserstein distance and in general optimal transport as a loss function and an optimization mechanism in deep learning.We look at uses of wasserstein distance in generative problems such as generative adversial networks, variational auto-encoders and diffusion models. We also perform experiments comparing the methods, on our problem of super-resolution of satellite images in africa 
\end{abstract}
\section{Introduction}
\section{Background}
\subsection{Optimal Transport}
\subsection{Wasserstein Distance}
\subsubsection{Definition}
\subsubsection{Primal Dual Formulation}
\section{Wasserstein Distance for GANs}
\subsection{Loss function formulation}
\subsection{Approximation and Justification}
\subsection{Other extensions}
\section{Wasserstein Distance for VAEs}
\subsection{Loss function formulation}
\subsection{Other extensions}
\section{Wasserstein Distance for Diffusion Models}

\subsection{Loss function formulation}
\subsection{Approximation and Justification}
\subsection{Other extensions}
\section{Neural Optimal Transport}
\subsection{Extension - Kernel Neural Optimal Transport}
\section{Application and Comparisons}
\subsection{Problem Definition}
\subsection{Dataset}
\subsection{Results and Discussion}
\section{Conclusion}
\section*{References}


References follow the acknowledgments in the camera-ready paper. Use unnumbered first-level heading for
the references. Any choice of citation style is acceptable as long as you are
consistent. It is permissible to reduce the font size to \verb+small+ (9 point)
when listing the references.
Note that the Reference section does not count towards the page limit.
\medskip


{
\small


[1] Alexander, J.A.\ \& Mozer, M.C.\ (1995) Template-based algorithms for
connectionist rule extraction. In G.\ Tesauro, D.S.\ Touretzky and T.K.\ Leen
(eds.), {\it Advances in Neural Information Processing Systems 7},
pp.\ 609--616. Cambridge, MA: MIT Press.


[2] Bower, J.M.\ \& Beeman, D.\ (1995) {\it The Book of GENESIS: Exploring
  Realistic Neural Models with the GEneral NEural SImulation System.}  New York:
TELOS/Springer--Verlag.


[3] Hasselmo, M.E., Schnell, E.\ \& Barkai, E.\ (1995) Dynamics of learning and
recall at excitatory recurrent synapses and cholinergic modulation in rat
hippocampal region CA3. {\it Journal of Neuroscience} {\bf 15}(7):5249-5262.
}

%%%%%%%%%%%%%%%%%%%%%%%%%%%%%%%%%%%%%%%%%%%%%%%%%%%%%%%%%%%%


\end{document}